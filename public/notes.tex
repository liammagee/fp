\documentclass[]{article}
\usepackage{lmodern}
\usepackage{amssymb,amsmath}
\usepackage{ifxetex,ifluatex}
\usepackage{fixltx2e} % provides \textsubscript
\ifnum 0\ifxetex 1\fi\ifluatex 1\fi=0 % if pdftex
  \usepackage[T1]{fontenc}
  \usepackage[utf8]{inputenc}
\else % if luatex or xelatex
  \ifxetex
    \usepackage{mathspec}
    \usepackage{xltxtra,xunicode}
  \else
    \usepackage{fontspec}
  \fi
  \defaultfontfeatures{Mapping=tex-text,Scale=MatchLowercase}
  \newcommand{\euro}{€}
\fi
% use upquote if available, for straight quotes in verbatim environments
\IfFileExists{upquote.sty}{\usepackage{upquote}}{}
% use microtype if available
\IfFileExists{microtype.sty}{%
\usepackage{microtype}
\UseMicrotypeSet[protrusion]{basicmath} % disable protrusion for tt fonts
}{}
\usepackage{color}
\usepackage{fancyvrb}
\newcommand{\VerbBar}{|}
\newcommand{\VERB}{\Verb[commandchars=\\\{\}]}
\DefineVerbatimEnvironment{Highlighting}{Verbatim}{commandchars=\\\{\}}
% Add ',fontsize=\small' for more characters per line
\newenvironment{Shaded}{}{}
\newcommand{\KeywordTok}[1]{\textcolor[rgb]{0.00,0.44,0.13}{\textbf{{#1}}}}
\newcommand{\DataTypeTok}[1]{\textcolor[rgb]{0.56,0.13,0.00}{{#1}}}
\newcommand{\DecValTok}[1]{\textcolor[rgb]{0.25,0.63,0.44}{{#1}}}
\newcommand{\BaseNTok}[1]{\textcolor[rgb]{0.25,0.63,0.44}{{#1}}}
\newcommand{\FloatTok}[1]{\textcolor[rgb]{0.25,0.63,0.44}{{#1}}}
\newcommand{\CharTok}[1]{\textcolor[rgb]{0.25,0.44,0.63}{{#1}}}
\newcommand{\StringTok}[1]{\textcolor[rgb]{0.25,0.44,0.63}{{#1}}}
\newcommand{\CommentTok}[1]{\textcolor[rgb]{0.38,0.63,0.69}{\textit{{#1}}}}
\newcommand{\OtherTok}[1]{\textcolor[rgb]{0.00,0.44,0.13}{{#1}}}
\newcommand{\AlertTok}[1]{\textcolor[rgb]{1.00,0.00,0.00}{\textbf{{#1}}}}
\newcommand{\FunctionTok}[1]{\textcolor[rgb]{0.02,0.16,0.49}{{#1}}}
\newcommand{\RegionMarkerTok}[1]{{#1}}
\newcommand{\ErrorTok}[1]{\textcolor[rgb]{1.00,0.00,0.00}{\textbf{{#1}}}}
\newcommand{\NormalTok}[1]{{#1}}
\usepackage{longtable,booktabs}
\ifxetex
  \usepackage[setpagesize=false, % page size defined by xetex
              unicode=false, % unicode breaks when used with xetex
              xetex]{hyperref}
\else
  \usepackage[unicode=true]{hyperref}
\fi
\hypersetup{breaklinks=true,
            bookmarks=true,
            pdfauthor={Liam Magee},
            pdftitle={Fierce Planet Redesign},
            colorlinks=true,
            citecolor=blue,
            urlcolor=blue,
            linkcolor=magenta,
            pdfborder={0 0 0}}
\urlstyle{same}  % don't use monospace font for urls
\setlength{\parindent}{0pt}
\setlength{\parskip}{6pt plus 2pt minus 1pt}
\setlength{\emergencystretch}{3em}  % prevent overfull lines
\setcounter{secnumdepth}{0}

\title{Fierce Planet Redesign}
\author{Liam Magee}
\date{20 June, 2014}

\begin{document}
\maketitle

{
\hypersetup{linkcolor=black}
\setcounter{tocdepth}{2}
\tableofcontents
}
\section{Overview}\label{overview}

\section{Components}\label{components}

\subsection{World}\label{world}

\subsubsection{Time}\label{time}

\begin{itemize}
\itemsep1pt\parskip0pt\parsep0pt
\item
  \emph{Origin time} (\(t_0\))
\item
  \emph{Time increment} (\(i\))
\item
  \emph{End time} (\(t_n\))
\item
  \emph{Current time} (\(t_i = t_0 + i\))
\end{itemize}

\emph{Origin time} defaults to \textbf{0}; \emph{Time increment} to
\textbf{1}; \emph{End time} to \textbf{infinity}.

\subsubsection{Space}\label{space}

Uses a grid structure with:

\begin{itemize}
\itemsep1pt\parskip0pt\parsep0pt
\item
  \emph{Width}
\item
  \emph{Height}
\item
  \emph{Cell} (depends on cell size)
\item
  \emph{Terrain}
\end{itemize}

\paragraph{Notes:}\label{notes}

\begin{itemize}
\itemsep1pt\parskip0pt\parsep0pt
\item
  Terrain extracted from SRTM (see addendum below).
\item
  \href{http://dds.cr.usgs.gov/srtm/version2_1/Documentation/Quickstart.pdf}{SRTM3
  Dimensions}

  \begin{itemize}
  \itemsep1pt\parskip0pt\parsep0pt
  \item
    Each tile is a 3-arc second
  \item
    1201x1201 pixels
  \item
    Each pixel is 90m (tile = 108,090m = 108km\textsuperscript{2})
  \end{itemize}
\item
  DECISION: Needs to re-rendered in world as 4km\textsuperscript{2}? Too
  big, the world is unnavigable (and unrenderable - though check with
  Unreal)
\end{itemize}

\subsection{Cell}\label{cell}

\begin{itemize}
\itemsep1pt\parskip0pt\parsep0pt
\item
  Each cell to represent a square (rectangular?) habitable `block'.
  Approx 100m\textsuperscript{2}. (Consult urban planning literature for
  more details -
  \href{http://en.wikipedia.org/wiki/Urban_planning}{insula}).
\item
  World therefore has 1600 patchs (= (4000 / 100)\textsuperscript{2}).
\item
  Cells have a series of heightmaps (corresponding to imported terrain).
\item
  If average heightmap is below 1m, it is assumed to be \textbf{water}.
\item
  ASSUMPTION: For now, no other natural features (e.g.~trees)
\end{itemize}

Further notes:

\begin{itemize}
\itemsep1pt\parskip0pt\parsep0pt
\item
  See Unreal / Unity docs on terrain for better terminology.
\end{itemize}

\subsection{Agents}\label{agents}

\begin{itemize}
\itemsep1pt\parskip0pt\parsep0pt
\item
  A \textbf{World} has \(\{0,...,m\}\) Agents
\item
  Agents are (humanoid?) objects that exhibit (minimally):

  \begin{itemize}
  \itemsep1pt\parskip0pt\parsep0pt
  \item
    \hyperref[life]{Life}
  \item
    \hyperref[spatiality]{Spatiality}
  \item
    \hyperref[motility]{Motility}
  \end{itemize}
\item
  Might also have:

  \begin{itemize}
  \itemsep1pt\parskip0pt\parsep0pt
  \item
    \hyperref[vitality]{Vitality}
  \item
    \hyperref[relationality]{Relationality}
  \item
    \hyperref[physicality]{Physicality}
  \item
    \hyperref[mind]{Mind}

    \begin{itemize}
    \itemsep1pt\parskip0pt\parsep0pt
    \item
      \hyperref[cognition]{Cognition}
    \item
      \hyperref[affect]{Affect}
    \item
      \hyperref[conation]{Conation}
    \end{itemize}
  \end{itemize}
\end{itemize}

\hyperdef{}{life}{\subsubsection{Life}\label{life}}

\begin{itemize}
\itemsep1pt\parskip0pt\parsep0pt
\item
  \emph{Birth}: year (month and day?) or \(b = t_x\)
\item
  \emph{Death}: year (month and day?) or \(d = t_y\)
\item
  \emph{Life elapsed} (\(d - b\))
\item
  \emph{Alive} (\(t_i > b \&\& t_i < d\))
\end{itemize}

\hyperdef{}{spatiality}{\subsubsection{Spatiality}\label{spatiality}}

For rendering purposes, agents have three-dimensional spatial extension.

\begin{itemize}
\itemsep1pt\parskip0pt\parsep0pt
\item
  \emph{Position}
\item
  \emph{Rotation}
\item
  \emph{Scale}
\end{itemize}

In common rendering environments, these are represented by:

\begin{longtable}[c]{@{}ll@{}}
\toprule
\begin{minipage}[b]{0.29\columnwidth}\raggedright\strut
Unreal Editor
\strut\end{minipage} &
\begin{minipage}[b]{0.22\columnwidth}\raggedright\strut
Unity 4
\strut\end{minipage}\tabularnewline
\midrule
\endhead
\begin{minipage}[t]{0.29\columnwidth}\raggedright\strut
Position Rotation Scale
\strut\end{minipage} &
\begin{minipage}[t]{0.22\columnwidth}\raggedright\strut
Position Rotation Scale
\strut\end{minipage}\tabularnewline
\bottomrule
\end{longtable}

This should further relate to \hyperref[place]{place}.

\hyperdef{}{motility}{\subsubsection{Motility}\label{motility}}

\textbf{Notes:}

\begin{itemize}
\itemsep1pt\parskip0pt\parsep0pt
\item
  Spontaneous movement (with or without a Goal)
\item
  Movement is in 3 dimensional space (has X, Y, Z co-ordinates)
\item
  Examine \emph{direct} (XYZ-transforms) vs \emph{indirect} (using game
  engine physics, forces - preferred but semi-deterministic?)
\end{itemize}

TODO: - Examine AI movement algoriths (e.g.~A\^{}*\^{} in
\textbf{Unity})

\hyperdef{}{relationality}{\subsubsection{Relationality}\label{relationality}}

\begin{itemize}
\itemsep1pt\parskip0pt\parsep0pt
\item
  \emph{Mother} (\(mother\))
\item
  \emph{Father} (\(father\))
\item
  \emph{Children} (\(children_{1..m}\))
\item
  \emph{Friends} (\(friends_{1..n}\))
\end{itemize}

Consider 1st class \emph{Relation}: - Kinds of relations: + Bonding +
Bridging - Type (one of: \{familial, professional, social\}?) - Strength
- Duration - Frequency

{[}NOTE: \textbf{Just} modelling relations would be of interest.{]}

\hyperdef{}{mind}{\subsubsection{Mind}\label{mind}}

\emph{An Overview of the Conative Domain} (Huitt and Cain 2005):

\begin{quote}
Psychology has traditionally identified and studied three components of
mind: cognition, affect, and conation (Hilgard, 1980; Huitt, 1996;
Tallon, 1997).
\end{quote}

\begin{quote}
Cognition refers to the process of coming to know and understand; of
encoding, perceiving, storing, processing, and retrieving information.
It is generally associated with the question of ``what'' (e.g., what
happened, what is going on now, what is the meaning of that
information.)
\end{quote}

\begin{quote}
Affect refers to the emotional interpretation of perceptions,
information, or knowledge. It is generally associated with one's
attachment (positive or negative) to people, objects, ideas, etc. and is
associated with the question ``How do I feel about this knowledge or
information?''
\end{quote}

\begin{quote}
Conation refers to the connection of knowledge and affect to behavior
and is associated with the issue of ``why.'' It is the personal,
intentional, planful, deliberate, goal-oriented, or striving component
of motivation, the proactive (as opposed to reactive or habitual) aspect
of behavior (Baumeister, Bratslavsky, Muraven \& Tice, 1998; Emmons,
1986). Atman (1987) defined conation as ``vectored energy: i.e.,
personal energy that has both direction and magnitude'' (p.~15). It is
closely associated with the concepts of intrinsic motivation, volition,
agency, self-direction, and self-regulation (Kane, 1985; Mischel, 1996).
\end{quote}

\begin{itemize}
\itemsep1pt\parskip0pt\parsep0pt
\item
  Relates to behaviour and action?
\end{itemize}

\hyperdef{}{cognition}{\paragraph{Cognition / Belief}\label{cognition}}

Refers to the general possesion of \emph{cognition}.

\begin{itemize}
\itemsep1pt\parskip0pt\parsep0pt
\item
  \emph{Memory} (\(memories_{1..n}\))
\item
  \emph{Reasoning} {[}cf. \hyperref[ontology]{Ontology}{]}
\end{itemize}

In \emph{BDI} frameworks, this corresponds to \emph{Beliefs}?

TODO: \emph{Reasoning} implies separate logical thread for computation?

\subparagraph{Memory / Belief}\label{memory-belief}

\begin{itemize}
\itemsep1pt\parskip0pt\parsep0pt
\item
  \emph{Subject}
\item
  \emph{Predicate}
\item
  \emph{Object}. Can be one of:

  \begin{itemize}
  \itemsep1pt\parskip0pt\parsep0pt
  \item
    \emph{Agent}
  \item
    \emph{Place}
  \end{itemize}
\item
  \emph{Date}
\item
  \emph{Strength} of memory or belief
\end{itemize}

\paragraph{Affectivity / Desire}\label{affectivity}

Refers to the emotional affects experienced. In \emph{BDI} frameworks,
this corresponds to \emph{Desires}.

\begin{itemize}
\itemsep1pt\parskip0pt\parsep0pt
\item
  \emph{Affect} (from range of emotional states?)
\end{itemize}

In \emph{BDI} frameworks, this corresponds to \emph{Desires}?

\hyperdef{}{conation}{\paragraph{Conation / Intention}\label{conation}}

Refers to the proactive intention to accomplish a \emph{Goal}, based on
a series of planned \emph{Actions}, to satisfy a \emph{Desire} based on
current \emph{Beliefs}.

\begin{itemize}
\itemsep1pt\parskip0pt\parsep0pt
\item
  \emph{Plans} (from range of emotional states?)
\end{itemize}

In \emph{BDI} frameworks, this corresponds to \emph{Intentions} (or both
\emph{Desires} and \emph{Intentions})?

\subparagraph{Plan}\label{plan}

\begin{itemize}
\itemsep1pt\parskip0pt\parsep0pt
\item
  \emph{Goal}
\item
  \emph{Action} (\(actions_{1..n}\))
\end{itemize}

\subparagraph{Goal}\label{goal}

Achieved when:

\begin{itemize}
\itemsep1pt\parskip0pt\parsep0pt
\item
  Some change in the world is effected

  \begin{itemize}
  \itemsep1pt\parskip0pt\parsep0pt
  \item
    Change in state, position\ldots{} of an \emph{Object}
  \end{itemize}
\item
  Some \emph{Belief} is modified?
\item
  Some \emph{Desire} is satisfied
\item
  Some \emph{Intention} is carried out
\end{itemize}

\subparagraph{Action}\label{action}

One of:

\begin{itemize}
\item
  \emph{Movement Act}
\item
  \emph{Speech Act}
\item
  Exercises a \emph{Capability}
\end{itemize}

Movement Act

\hyperdef{}{physicality}{\subsubsection{Physicality}\label{physicality}}

Includes: - \emph{Shape} - \emph{Color}

TODO: UNSPECIFIED - relate to Unreal/ Unity properties

\begin{longtable}[c]{@{}ll@{}}
\toprule
\begin{minipage}[b]{0.29\columnwidth}\raggedright\strut
Unreal Editor
\strut\end{minipage} &
\begin{minipage}[b]{0.22\columnwidth}\raggedright\strut
Unity 4
\strut\end{minipage}\tabularnewline
\midrule
\endhead
\begin{minipage}[t]{0.29\columnwidth}\raggedright\strut
TBD
\strut\end{minipage} &
\begin{minipage}[t]{0.22\columnwidth}\raggedright\strut
\strut\end{minipage}\tabularnewline
\bottomrule
\end{longtable}

\subsection{Social Systems}\label{socialux5fsystem}

\subsubsection{Institutions}\label{institutions}

\begin{itemize}
\itemsep1pt\parskip0pt\parsep0pt
\item
  Law

  \begin{itemize}
  \itemsep1pt\parskip0pt\parsep0pt
  \item
    Legislative
  \item
    Judicial
  \item
    Executive
  \end{itemize}
\item
  Economy
\item
  Governance
\item
  Knowledge

  \begin{itemize}
  \itemsep1pt\parskip0pt\parsep0pt
  \item
    Science
  \item
    Culture
  \end{itemize}
\end{itemize}

\paragraph{Legal}\label{legal}

Wikipedia links: -
\href{http://en.wikipedia.org/wiki/Legal_doctrine}{Legal doctrine} -
\href{http://en.wikipedia.org/wiki/Constitutionalism}{Constitutionalism}

\subsubsection{Technoscience}\label{technoscience}

\paragraph{Explaining technoscience as a
concept}\label{explaining-technoscience-as-a-concept}

\paragraph{Categories}\label{categories}

\begin{itemize}
\item
  Power
\item
  Building
\item
\end{itemize}

\paragraph{History}\label{history}

\begin{itemize}
\item
  Steam power
\item
\end{itemize}

\hyperdef{}{place}{\subsection{Place}\label{place}}

\begin{itemize}
\itemsep1pt\parskip0pt\parsep0pt
\item
  \emph{Name}
\item
  \emph{Location}
\item
  \emph{Extent}

  \begin{itemize}
  \itemsep1pt\parskip0pt\parsep0pt
  \item
    \emph{Width}
  \item
    \emph{Height}
  \end{itemize}
\item
  \emph{Designation}, one of \{\emph{Home}, \emph{Neighbourhood},
  \emph{City} or \emph{District}, \emph{Region}, \emph{Country}\}
\end{itemize}

\subsection{Cultures}\label{culture}

\begin{itemize}
\itemsep1pt\parskip0pt\parsep0pt
\item
  \emph{Name}
\item
  Set of \emph{Capabilities} (\(capabilities_{1..n}\))
\end{itemize}

\subsection{Institutions}\label{institutions}

\begin{itemize}
\itemsep1pt\parskip0pt\parsep0pt
\item
  \emph{Name}
\item
  \emph{Location}
\item
  \emph{Type} (one of: )
\end{itemize}

\subsection{Built environment}\label{built-environment}

\begin{itemize}
\itemsep1pt\parskip0pt\parsep0pt
\item
  Patches can have \(\{0,...,n\}\) Buildings.
\item
  Patches can have Roads, Alleys, Pavements
\end{itemize}

\subsubsection{Urban morphology}\label{urban-morphology}

\paragraph{Links}\label{links}

Wikipedia:

\begin{itemize}
\itemsep1pt\parskip0pt\parsep0pt
\item
  \href{http://en.wikipedia.org/wiki/Urban_morphology}{Wikipedia:Urban
  Morphology}
\item
  \href{http://en.wikipedia.org/wiki/Urban_area}{Wikipedia:Urban area}
\item
  \href{http://en.wikipedia.org/wiki/Global_City}{Global City}
\item
  \href{http://en.wikipedia.org/wiki/Index_of_urban_studies_articles}{Index
  of urban studies articles}
\end{itemize}

General:

\begin{itemize}
\itemsep1pt\parskip0pt\parsep0pt
\item
  \href{https://www.google.com.co/search?q=urban+morphology\&espv=2\&biw=1018\&bih=557\&tbm=isch\&tbo=u\&source=univ\&sa=X\&ei=aEavU_2GE8Ss0QXk-oDoAg\&ved=0CDUQsAQ}{Google
  Images}
\item
  \href{http://www.urbanmorphologyinstitute.org/}{Urban Morphology \&
  Complex Systems Institute}

  \begin{itemize}
  \itemsep1pt\parskip0pt\parsep0pt
  \item
    \href{http://www.urbanmorphologyinstitute.org/research/papers/}{Papers}
  \end{itemize}
\item
  \href{http://urbanmorphology.org/}{International Seminar on Urban Form
  / Urban Morphology Journal}
\item
  \href{http://emergenturbanism.com/}{Emergent Urbanism}
\item
  \href{http://historiccitiesrules.com/}{Historic Cities Rules}
\item
  \href{http://www.patternlanguage.com/}{Pattern Language}
\item
  \href{http://carfree.com/}{Carfree}
\item
  \href{http://www.humantransit.org/}{Human Transit}
\item
  \href{http://katarxis3.com/}{Katarxis}
\item
  \href{http://intbau.org/}{International Network for Traditional
  Building, Architecture \& Urbanism}
\item
  \href{http://kunstlercast.com/}{KunstlerCast}
\item
  \href{http://zeta.math.utsa.edu/~yxk833/}{Nikos A. Salingaros}
\item
  \href{http://www.pps.org/}{Project for Public Spaces}
\item
  \href{http://wolframscience.com/}{Wolfram Science}
\item
  \href{http://codesproject.asu.edu/}{The Codes Project}
\item
  \href{http://www.esri.com/software/cityengine}{ESRI}
\item
  \href{http://www.spacesyntax.net/}{Space Syntax Network}
\item
  \href{http://leav.versailles.archi.fr/\#/equipes-de-recherche/axes-de-recherche-2014-2019/\%C3\%A9nergie-climat-environnement}{Laboratoire
  de Recherche del l'Ecole Nationale Superieure d'Architecture de
  Versailles}
\item
  \href{http://www.stadtbaukunst.org/english/institute/index.html}{Institut
  der Stadtbaukunst (A Forum for Architecture and Urban Design)}

  \begin{itemize}
  \itemsep1pt\parskip0pt\parsep0pt
  \item
    \href{http://www.stadtbaukunst.org/english/texts-about-urban-design/index.html}{Publications}
  \end{itemize}
\item
  \href{http://www.newurbanism.org/}{New Urbanism}
\item
  Amazon

  \begin{itemize}
  \itemsep1pt\parskip0pt\parsep0pt
  \item
    Serge Salat:
    \href{http://www.amazon.com/Cities-Forms-On-Sustainable-Urbanism/dp/2705681116}{Cities
    and Forms: On Sustainable Urbanism}
  \item
    Spiro Kostof:
    \href{http://www.amazon.com/The-City-Assembled-Elements-Through/dp/0821219308/ref=pd_sim_b_1?ie=UTF8\&refRID=19Z04ZAV2K0Z7784G86A}{The
    City Assembled: The Elements of Urban Form Through History}
  \item
    Lydia Otero:
    \href{http://www.amazon.com/La-Calle-Spatial-Conflicts-Southwest/dp/0816528888/ref=pd_sim_b_4?ie=UTF8\&refRID=0GDA25RHFDPF2QQDMR0C}{La
    Calle: Spatial Conflicts and Urban Renewal in a Southwest City}
  \item
    \href{http://www.amazon.com/Battle-Life-Beauty-Earth-World-Systems/dp/0199898073/ref=la_B000AQ4JVU_1_3?ie=UTF8\&qid=1369782219\&sr=1-3}{The
    Battle for the Life and Beauty of the Earth: A Struggle Between Two
    World-Systems}
  \item
    \href{http://www.amazon.com/Carfree-Design-Manual-J-Crawford/dp/9057270609/ref=pd_sim_b_1?ie=UTF8\&refRID=0R5CVQM8RRWQ5SM5QDM1}{Carfree
    Design Manual}
  \item
    \href{http://www.amazon.com/exec/obidos/ASIN/9057270420/}{Carfree
    Cities}
  \item
    \href{http://www.amazon.com/Human-Transit-Clearer-Thinking-Communities-ebook/dp/B008LVR1KM/ref=tmm_kin_title_0?ie=UTF8\&qid=1402344064\&sr=1-1}{Human
    Transit: How Clearer Thinking about Public Transit Can Enrich Our
    Communities and Our Lives}
  \end{itemize}
\end{itemize}

Authors:

\begin{itemize}
\itemsep1pt\parskip0pt\parsep0pt
\item
  Lewis Mumford
\item
  Peter Hall
\item
  Michael Batty
\item
  Saskia Sassen
\item
  Jan Gehl
\item
  Serge Salat
\item
  Nikos A. Salingaros
\item
  Spiro Kostof
\item
  Patrick Geddes
\end{itemize}

\paragraph{Links}\label{links-1}

\begin{itemize}
\itemsep1pt\parskip0pt\parsep0pt
\item
  \href{http://vimeo.com/88794901}{Houdini Engine for Unity}
\item
  \href{http://vterrain.org/Culture/BldCity/Proc/}{Procedural Building
  Implementations}
\end{itemize}

\paragraph{Procedural Building}\label{procedural-building}

\begin{itemize}
\itemsep1pt\parskip0pt\parsep0pt
\item
  ASSUMPTION: Buildings should be procedurally generated.
\end{itemize}

For \texttt{Fierce\ Planet}, procedural building should develop a
pseudo-realistic urban environment.

Procedure:

\begin{itemize}
\itemsep1pt\parskip0pt\parsep0pt
\item
  Has floor, ceiling, walls, levels (staircases between them)
\item
  Quality of construction
\item
  Rendering (textures, materials)
\end{itemize}

\subparagraph{v1}\label{v1}

We assume:

\begin{itemize}
\itemsep1pt\parskip0pt\parsep0pt
\item
  \(1..m\) motile \emph{Agents} (\(Agents\))
\item
  \(1..n\) \emph{Buildings} (\(Buildings\))
\item
  some \emph{World} with \emph{Terrain} with suitable heightmap
  characterstics for building
\item
  A \emph{Centre point} (\(Origin = {x, y}\)) (based on the origin of
  the world)
\item
  a series of rectangular \emph{Cells} (determined by \emph{World} patch
  size)
\item
  a \(WorldRadius\), the distance from the world to the boundaries
\end{itemize}

Then we calculate, for any given \(cell\), the likelihood of growing a
building at time \(t\):

\begin{itemize}
\itemsep1pt\parskip0pt\parsep0pt
\item
  Whether agents exist in the cell at \(t\):
  \(\sum(agents_{cell_t}) > 0\)
\item
  The Euclidean distance (\(RelativeEd\)) of the cell from the origin,
  relative to the boundary (\(WorldRadius\))
\item
  The building-to-agent ratio (\(baRatio\)) of the \emph{World}
\item
  The chance to build as a product of the Euclidean distance and 1 minus
  the ratio (\(ChanceToBuild = RelativeEd (1 - baRatio)\))
\end{itemize}

In Ruby pseudo-code:

\begin{Shaded}
\begin{Highlighting}[]

\NormalTok{agents_at_cell = cell.agents.length}
\KeywordTok{if} \NormalTok{(agents_at_cell > }\DecValTok{0}\NormalTok{) }
    \NormalTok{ed = (cell.x - origin.x + cell.y - origin.y).abs}
    \NormalTok{relative_ed = (euclidean_distance / world_radius).abs}
    \NormalTok{ba_ratio = buildings.length / agents.length}
    \NormalTok{chance_to_build = relative_ed * (}\DecValTok{1} \NormalTok{- ba_ratio)}
\KeywordTok{end}
\KeywordTok{if} \NormalTok{(rand() < chance_to_build)}
    \NormalTok{createBuilding()}
\KeywordTok{end}
\end{Highlighting}
\end{Shaded}

\subparagraph{v2: An agent-based procedural city
engine}\label{v2-an-agent-based-procedural-city-engine}

\begin{itemize}
\itemsep1pt\parskip0pt\parsep0pt
\item
  Focus on modularity - different urban forms \& morphologies
\item
  Types:

  \begin{itemize}
  \itemsep1pt\parskip0pt\parsep0pt
  \item
    Chaotic
  \item
    Geometric
  \end{itemize}
\item
  Principles:

  \begin{itemize}
  \itemsep1pt\parskip0pt\parsep0pt
  \item
    Respect geography:

    \begin{itemize}
    \itemsep1pt\parskip0pt\parsep0pt
    \item
      Terrain gradient
    \item
      Sea level
    \item
      Ground surface - habitable status indicator?

      \begin{itemize}
      \itemsep1pt\parskip0pt\parsep0pt
      \item
        Rock
      \item
        Forest
      \item
        Field
      \item
        Marsh
      \item
        Wetlands
      \item
        Desert
      \end{itemize}
    \item
      Proximity to necessities:

      \begin{itemize}
      \itemsep1pt\parskip0pt\parsep0pt
      \item
        Water
      \item
        Food
      \end{itemize}
    \end{itemize}
  \item
    Construction technologies

    \begin{itemize}
    \itemsep1pt\parskip0pt\parsep0pt
    \item
      Materials

      \begin{itemize}
      \itemsep1pt\parskip0pt\parsep0pt
      \item
        Timber (and quality)
      \item
        Metals
      \end{itemize}
    \item
      Tools
    \item
      Know-how (design)
    \item
      Cultural preferences
    \end{itemize}
  \end{itemize}
\item
  Agent activity

  \begin{itemize}
  \itemsep1pt\parskip0pt\parsep0pt
  \item
    Repeated traversal: creates pathways, trails, roads
  \item
    Evolutionary movement - improved strategies
  \end{itemize}
\item
  Vehicles - technology
\end{itemize}

References

\begin{itemize}
\itemsep1pt\parskip0pt\parsep0pt
\item
  \href{http://www.shamusyoung.com/twentysidedtale/?p=2940}{Procedural
  City, Part 1: Introduction}
\item
  \href{http://learningthreejs.com/blog/2013/08/02/how-to-do-a-procedural-city-in-100lines/}{How
  to Do a Procedural City in 100 Lines}
\item
  \href{http://graphics.ethz.ch/Downloads/Publications/Papers/2001/p_Par01.pdf}{Procedural
  Modeling of Cities}
\item
  \href{http://www.vision.ee.ethz.ch/~pmueller/documents/procedural_modeling_of_cities__siggraph2001.pdf}{}
\end{itemize}

\subsection{Statistics}\label{statistics}

Minimally must consider:

\begin{itemize}
\itemsep1pt\parskip0pt\parsep0pt
\item
  Net Migration

  \begin{itemize}
  \itemsep1pt\parskip0pt\parsep0pt
  \item
    Immigration
  \item
    Emigration
  \end{itemize}
\item
  Life expectancy
\item
  Fertility rates
\item
  Child morbidity
\end{itemize}

Also consider DBPedia sources (TODO: what are common spatial fields?).

\subsubsection{Indicators}\label{indicators}

\begin{itemize}
\itemsep1pt\parskip0pt\parsep0pt
\item
  \href{http://en.wikipedia.org/wiki/Global_City}{Global City}

  \begin{itemize}
  \itemsep1pt\parskip0pt\parsep0pt
  \item
    GaWC
  \item
    Global Cities Index
  \item
    Global Economic Power Index
  \item
    Global Power City Index
  \item
    The Wealth Report
  \item
    Global City Competitiveness Index
  \end{itemize}
\end{itemize}

\paragraph{Landscape}\label{landscape}

\begin{itemize}
\itemsep1pt\parskip0pt\parsep0pt
\item
  \href{https://docs.unrealengine.com/latest/INT/Engine/Landscape/Materials/index.html\#landscape-specificmaterialnodes}{Landscape
  Materials}
\item
  \href{https://forums.epicgames.com/threads/915737-Landscape-and-Textures}{Landscape
  and Textures}
\item
  \href{https://wiki.unrealengine.com/Landscape_-_Sizes_and_Height_Guide}{Sizes
  and Height Guide}
\item
  \href{http://udn.epicgames.com/Three/TerrainAlphamaps.html}{Terrain
  Alphamaps}
\item
  \href{https://forums.unrealengine.com/showthread.php?3884-Heightmap-DTM-Landscape-Formats}{Heightmap
  DTM Landscape Formats}
\item
  \href{http://scrawkblog.com/2013/02/05/using-dem-data-to-create-height-maps/}{Using
  DEM data to create height maps}
\item
  \href{http://forum.world-machine.com/index.php?topic=1299.0}{Importing
  DEM into World Machine}
\item
  \href{http://www.world-machine.com/World\%20Machine\%20Help.pdf}{WORLD
  MACHINE USERS GUIDE}
\item
  \href{https://forums.unrealengine.com/showthread.php?1265-Tutorial-My-Workflow-for-Creating-Terrain}{My
  Workflow for Creating Terrain}
\item
  \href{https://www.youtube.com/watch?v=K9WTKK9f1b8}{Unreal Engine -
  Getting World Machine Terrains in your game!}
\item
  \href{https://www.youtube.com/watch?v=4SQPzgzAsfI}{Unreal Engine 4:
  Terrains with World Machine \& Material Function Tutorial}
\item
  \href{https://wiki.unrealengine.com/Using_Weightmaps_to_texture_a_Landscape}{Using
  Weightmaps to texture a Landscape}
\item
  \href{https://wiki.unrealengine.com/Landscape_Tool_-_Video}{Landscape
  Tool}
\item
  \href{http://www.world-machine.com/about.php?page=features}{World
  Machine}
\end{itemize}

\begin{verbatim}

gdal_translate -scale 0 2470 0 255 -outsize 200 200 -of PNG syd2.tif syd.png

gdal_translate -scale 0 400 0 65535 -ot UInt16 -outsize 1601 1601 -of ENVI syd2.tif syd2.bin
\end{verbatim}

\subsubsection{World Machine}\label{world-machine}

\subsection{Artwork}\label{artwork}

\begin{itemize}
\itemsep1pt\parskip0pt\parsep0pt
\item
  \href{http://gametextures.com}{Game Textures}
\end{itemize}

\subsection{Interface Design}\label{interface-design}

\subsubsection{Inputs}\label{inputs}

\paragraph{Setup inputs:}\label{setup-inputs}

\begin{itemize}
\itemsep1pt\parskip0pt\parsep0pt
\item
  Social

  \begin{itemize}
  \itemsep1pt\parskip0pt\parsep0pt
  \item
    Population
  \item
    Immigration / emigration / net migration rates
  \item
    Fertility rate
  \end{itemize}
\item
  Geography

  \begin{itemize}
  \itemsep1pt\parskip0pt\parsep0pt
  \item
    Terrain
  \item
    Settlement area (bounds)
  \end{itemize}
\end{itemize}

\paragraph{Simulation Actions:}\label{simulation-actions}

\begin{itemize}
\itemsep1pt\parskip0pt\parsep0pt
\item
  Load world
\item
  Play / Pause
\item
  Restart
\item
  Step
\end{itemize}

\paragraph{Viewing Actions:}\label{viewing-actions}

\begin{itemize}
\itemsep1pt\parskip0pt\parsep0pt
\item
  Zoom
\item
  Pan
\item
  View (Agent / Top / Isometric)
\end{itemize}

\paragraph{Visibility toggles:}\label{visibility-toggles}

\begin{itemize}
\itemsep1pt\parskip0pt\parsep0pt
\item
  Landscape
\item
  Agents
\item
  Agent trails
\item
  Buildings - built environment
\item
  Networks
\item
  Cells
\item
  Grid
\end{itemize}

\paragraph{Interactions:}\label{interactions}

\begin{itemize}
\itemsep1pt\parskip0pt\parsep0pt
\item
  Rate of Time
\end{itemize}

\paragraph{Built Environment:}\label{built-environment-1}

Components:

\begin{itemize}
\itemsep1pt\parskip0pt\parsep0pt
\item
  Stocks

  \begin{itemize}
  \itemsep1pt\parskip0pt\parsep0pt
  \item
    Building types:

    \begin{itemize}
    \itemsep1pt\parskip0pt\parsep0pt
    \item
      Housing
    \item
      Transport
    \item
      Environment
    \item
      Health
    \item
      Finance
    \item
      Commerce
    \item
      Governance
    \item
      Sport
    \item
      Education
    \item
      Culture
    \end{itemize}
  \end{itemize}
\item
  Flows (movement)

  \begin{itemize}
  \itemsep1pt\parskip0pt\parsep0pt
  \item
    Human transport

    \begin{itemize}
    \itemsep1pt\parskip0pt\parsep0pt
    \item
      Roads

      \begin{itemize}
      \itemsep1pt\parskip0pt\parsep0pt
      \item
        Lanes
      \item
        Streets
      \item
        Major roads
      \item
        Freeways
      \item
        Highways
      \end{itemize}
    \item
      Paths

      \begin{itemize}
      \itemsep1pt\parskip0pt\parsep0pt
      \item
        Walking
      \item
        Bike
      \end{itemize}
    \item
      Nature strips
    \item
      Gutters
    \item
      Drains
    \item
      Car parks
    \item
      Train lines
    \item
      Tram lines
    \item
      Busways
    \item
      Bridges
    \item
      Tunnels
    \item
      Overpasses
    \item
      Roundabouts
    \item
      Stops, stations
    \end{itemize}
  \item
    Other transport

    \begin{itemize}
    \itemsep1pt\parskip0pt\parsep0pt
    \item
      Waterways
    \item
      Drains
    \item
      Sewers
    \item
      Power lines
    \item
      Water pipes
    \item
      Telecommunication cables
    \end{itemize}
  \end{itemize}
\end{itemize}

Distinctions:

\begin{itemize}
\itemsep1pt\parskip0pt\parsep0pt
\item
  Ownership

  \begin{itemize}
  \itemsep1pt\parskip0pt\parsep0pt
  \item
    Public
  \item
    Private
  \end{itemize}
\item
  Conditions of construction

  \begin{itemize}
  \itemsep1pt\parskip0pt\parsep0pt
  \item
    Financial capital availability
  \item
    Technological capacity
  \item
    Demand
  \item
    Regulatory framework
  \item
    Legacy environment
  \end{itemize}
\end{itemize}

\subsubsection{Output}\label{output}

\begin{itemize}
\itemsep1pt\parskip0pt\parsep0pt
\item
  Number of Agents (Population)
\item
  Average life of agents (Longevity)
\item
  Population density
\item
  Immigration / emigration / net migration
\item
  Fertility rate
\item
  Size of network
\item
  Number of buildings (housing capacity?)
\end{itemize}

Hypothetical outputs:

\begin{itemize}
\itemsep1pt\parskip0pt\parsep0pt
\item
  Environmental sustainability
\item
  Diversity
\item
  Economic output
\item
  Political stability
\end{itemize}

\section{Addenda}\label{addenda}

\subsection{Generating this document}\label{generating-this-document}

To generate HTML of this guide:

\begin{verbatim}
pandoc -s -S --toc --toc-depth=2 --mathjax --bibliography=fp.bib --template=templates/bootstrap --css=../css/pandoc-bootstrap.css notes.md -o output/notes.html && open output/notes.html
\end{verbatim}

To generate a PDF of this guide:

\begin{verbatim}
pandoc -s -S --toc --bibliography=fp.bib notes.md -o output/notes.pdf && open output/notes.pdf
\end{verbatim}

\subsubsection{Relevant references}\label{relevant-references}

\begin{itemize}
\itemsep1pt\parskip0pt\parsep0pt
\item
  \href{http://en.wikibooks.org/wiki/LaTeX/Mathematics}{WikiBooks:LaTeX/Mathematics}
\item
  \href{http://johnmacfarlane.net/pandoc/README.html}{Pandoc}
\item
  \href{http://getbootstrap.com/}{Bootstrap}

  \begin{itemize}
  \itemsep1pt\parskip0pt\parsep0pt
  \item
    \href{http://getbootstrap.com/components/}{Components}
  \end{itemize}
\item
  \href{http://www.mathjax.org/}{MathJax}
\end{itemize}

General document editing links: -
\href{http://drz.ac/2013/01/03/blogging-with-math/}{Blogging with math}
-
\href{http://stackoverflow.com/questions/4425198/markdown-target-blank}{markdown
target=``\_blank''}

\subsection{Importing tile maps into 3D
Engines}\label{importing-tile-maps-into-3d-engines}

Reference:

\begin{itemize}
\itemsep1pt\parskip0pt\parsep0pt
\item
  \href{http://alastaira.wordpress.com/2013/11/12/importing-dem-terrain-heightmaps-for-unity-using-gdal/}{Importing
  DEM into Unity}
\item
  \href{http://forum.unity3d.com/threads/23851-importing-real-maps-(DEMs)-into-unity}{Unity
  Notes}
\end{itemize}

Download \texttt{GDAL} utilities:

\begin{itemize}
\itemsep1pt\parskip0pt\parsep0pt
\item
  \href{http://trac.osgeo.org/gdal/wiki/DownloadingGdalBinaries}{General}
\item
  \href{http://www.kyngchaos.com/software:frameworks}{Mac}
\item
  \href{http://www.gdal.org/gdal_translate.html}{GDAL Docs}
\item
  \href{https://github.com/dwtkns/gdal-cheat-sheet}{GDAL Cheat Sheet}
\end{itemize}

Download tile maps from:

\begin{itemize}
\itemsep1pt\parskip0pt\parsep0pt
\item
  \href{http://gdex.cr.usgs.gov/gdex/}{Global Data Explorer}
\item
  \href{http://dwtkns.com/srtm/}{SRTM Tile Mapper}
\item
  \href{http://dds.cr.usgs.gov/srtm/version2_1/SRTM3/Australia/}{USGS}
\item
  \href{http://ws.csiss.gmu.edu/DEMExplorer/}{DEM Explorer}
\end{itemize}

Merge multiple tiles (for Melbourne):

\begin{verbatim}
gdal_merge.py -o Melbourne.hgt S38E144.hgt S38E145.hgt S39E144.hgt S39E145.hgt
\end{verbatim}

Translate to a TIF (DEM format):

\begin{verbatim}
gdal_translate Melbourne.hgt Melbourne.tif
\end{verbatim}

Prepare for \texttt{Unity}:

\begin{verbatim}
gdal_translate -ot UInt16 -scale -of ENVI -outsize 1025 1025 Melbourne.hgt Melbourne.raw
\end{verbatim}

Miscellaneous, from the
\href{https://github.com/dwtkns/gdal-cheat-sheet}{GDAL Cheat Sheet}

\begin{verbatim}
gdal_merge.py -co "PHOTOMETRIC=RGB" -separate LC81690372014137LGN00_B{4,3,2}.tif -o LC81690372014137LGN00_rgb.tif

gdaldem hillshade -of PNG S34E151.tif hillshade.png

gdaldem color-relief S34E151.tif ramp.txt color-relief.tif
gdaldem color-relief -of PNG S34E151.tif ramp.txt color-relief.png


gdaldem slope S34E151.tif slope.tif 

gdaldem color-relief slope.tif ramp2.txt slopeshade.tif

gdaldem hillshade -of PNG slopeshade.tif hillshade.png
\end{verbatim}

\subsubsection{\texorpdfstring{Prepare for
\texttt{Unreal}}{Prepare for Unreal}}\label{prepare-for-unreal}

Swapping little-endian to big-endian in Ruby
\href{http://stackoverflow.com/questions/16077885/how-to-convert-to-big-endian-in-ruby}{from
stackoverflow}. Alse see
\href{http://www.ruby-doc.org/core-2.1.2/String.html\#method-i-unpack}{Ruby
String\#unpack reference}.

\begin{Shaded}
\begin{Highlighting}[]

\NormalTok{f = }\DataTypeTok{File}\NormalTok{.new(}\StringTok{"/Users/liam/Downloads/Melbourne.raw"}\NormalTok{)}
\NormalTok{bs = f.read()}
\NormalTok{bsr = bs.unpack(}\StringTok{"s<*"}\NormalTok{).pack(}\StringTok{"s>*"}\NormalTok{)}
\NormalTok{f2 = }\DataTypeTok{File}\NormalTok{.new(}\StringTok{"/Users/liam/Downloads/Melbourne2.raw"}\NormalTok{, }\StringTok{"w+"}\NormalTok{)}
\NormalTok{f2.write(bsr)}
\end{Highlighting}
\end{Shaded}

\subsubsection{General Terrain mapping
references}\label{terrainux5freferences}

\begin{itemize}
\itemsep1pt\parskip0pt\parsep0pt
\item
  \href{https://lta.cr.usgs.gov/SRTM2}{USGS Site: SRTM}

  \begin{itemize}
  \itemsep1pt\parskip0pt\parsep0pt
  \item
    \href{http://dds.cr.usgs.gov/srtm/version2_1/Documentation/Quickstart.pdf}{Quickstart}
  \item
    \href{http://en.wikipedia.org/wiki/Shuttle_Radar_Topography_Mission}{Wikipedia}
  \end{itemize}
\item
  \href{http://www.ga.gov.au/metadata-gateway/metadata/record/gcat_66006}{Example
  data}
\item
  \href{http://www.webgis.com/srtm3.html}{WebGIS}
\item
  \href{http://dds.cr.usgs.gov/srtm/version2_1/Documentation/Quickstart.pdf}{Quickstart
  Guide}
\item
  Miscellaneous

  \begin{itemize}
  \itemsep1pt\parskip0pt\parsep0pt
  \item
    \url{http://stackoverflow.com/questions/357415/how-to-read-nasa-hgt-binary-files}
  \item
    \url{http://www.soi.city.ac.uk/~jwo/landserf/landserf230/doc/howto/srtm.html}
  \item
    \url{http://lists.osgeo.org/pipermail/gdal-dev/2011-November/030817.html}
  \item
    \url{http://vterrain.org/Elevation/global.html}
  \end{itemize}
\item
  \href{http://www.world-machine.com/}{World Machine}
\end{itemize}

\subsection*{Bibliography}\label{bibliography}
\addcontentsline{toc}{subsection}{Bibliography}

Huitt, William, and S Cain. 2005. ``An Overview of the Conative
Domain.'' \emph{Educational Psychology Interactive}, 1--20.

\end{document}
